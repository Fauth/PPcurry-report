% !TeX root = ./rapport.tex
\chapter{Cahier des charges et organisation}

\paragraph{}Le but est de développer un logiciel permettant la saisie de circuits électriques ainsi que la simulation de leur comportement. Il n'est pas attendu un logiciel de qualité professionnelle devant être effectivement utilisé : nous nous limitons donc à des fonctionnalités d'édition basiques (création, sauvegarde et chargement de circuits de taille limitée), avec une liste de composants disponibles réduite (source de tension alternative ou continue, source de courant, résistance, inductance, condensateur, fil et terre) et un fonctionnement de la simulation limité à des cas simples. Il n'y a pas de liste spécifique de capacités à remplir, mais nous avons une année pour faire du mieux possible.

\paragraph{}Le professeur encadrant nous ayant laissé une grande latitude organisationnelle, nous avons décidé d'organiser ce projet en deux grandes phases : tout d'abord, nous allons nous occuper du front-end, avec les capacités d'édition du circuit, puis du back-end, pour permettre les simulations. Cette première partie nous a pris 2 séquences scolaires sur 4 (le Memory en avait déjà occupé une), de début décembre à début mars. La dernière séquence sera donc allouée à la simulation, avec une approche itérative : l'algorithme sera complété au fur et à mesure afin de permettre de simuler des cas plus complexes avec des résultats plus précis.